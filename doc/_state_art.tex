\chapter{State of the art}

The landscape of Linux is separated by what is called as
``Linux distribution''. A Linux distribution (or distro for
short) is a collection of packages that are distributed
together as a part of the project. A package is a software
project that is distributed as a single unit, and contains
multiple files in the disk. The packages vary from the most
basic libraries and programs, like libc or bash, to any
complex software, like Google Chrome -- and also the Linux
kernel itself along with its kernel modules.

Linux packages are not meant to be shared between different
distribution. When a package is distributed for a specific
distro -- for example, Ubuntu -- it relies on libraries
provided by other Ubuntu packages. If we try to install this
package into a different distribution, for example Fedora,
it may or may not work. This problem is aggravated by the
fact that Linux libraries are very ``split'', as following
the UNIX philosophy of ``do one thing and do it well'', the
state of Linux distributions is that there are many
libraries that combine. The vast majority of these library
are written in C and C++, making Linux package managers the
de facto package manager for these languages
\cite{amor-iglesiasMeasuringLibreSoftware2005} . This is in
contrast to
other operating systems like Windows, where a package author
can rely on a ``base'' system that doesn't change, and can
bundle its own dependencies as part of the package, without
intermediate links.

Because each distribution has its own organization of
packages versions and customization, that are not
interoperable between different distributions, the Linux
ecosystem is very fragmented
\cite{espePerformanceEvaluationContainer2020}, even existing
distributions for specific applications
\cite{nemotoLin4NeuroCustomizedLinux2011} . The projects
use their own
package manager to install every package, as installing a
different package manager is not guaranteed to work. For
this reason, each distribution is only associated with its
reference package manager, being two sides of the same coin.
Some examples are:

\begin{itemize}
    \item Debian --- apt
    \item Fedora --- dnf
    \item Gentoo --- portage
    \item Arch --- pacman
\end{itemize}

With the increasing usage of internet services, Linux has
placed itself as the de facto operating system for servers.
As the online infrastructure of the world runs on Linux,
there is an increasing amount of developers that need to
deploy their applications into cloud services. One problem
that developers face is that applications are not portable
enough between different Linux distributions. This is partly
a result of this fragmentation that was mentioned earlier.
Updates of the same distribution can also cause
incompatibilities from one version to another.

As a result, an alternative software deployment solution to
just building and running packages in the host operating
system was developed by the company \textit{Docker}, and
later established in 2015 under the Linux Foundation as the
\textit{Open Container Initiative}. This solution uses the
concept of ``containers'', which are a collection of files
that run in a sandboxed environment from the host operating
system. This environment is usually based on some Linux
distribution such as Debian, but with the advantage that a
container may be built from any host \ac{OS} and deployed
into any host \ac{OS} .

Containers serve as a ``compatibility layer'' for classical
Linux distributions, as it doesn't change the underlying
functionality of packages themselves. But it is effective
enough to provide a good developer experience.

Earlier on the timeline, \citeauthor{dolstraNixOS2008}
researched this same problem of the classical software
deployment, and proposed a solution based on the concept of
a functional package manager \cite{dolstraNixOS2008} . After
years of development, this project has evolved into the Nix
package manager and the NixOS Linux distribution. Nix tries
an alternative approach to the classical package managers,
by prefixing each package with a unique path based on a hash
of its definition and dependencies. This allows multiple
packages to coexist in the same host operating system, and
also to use Nix on different host systems that are not
NixOS. While this solves the problem of reproducible
software deployments that Docker / OCI solves, the latter
also provides an application runtime which is used to
isolate the application from the host operating system \cite{espePerformanceEvaluationContainer2020}. In
this regard, Nix is just a system to build packages. One of
the problem that Nix faces is the usage of its own
language Nix (with the same name), that is based on the
concept of functional programming and pure function
application, with the inspiration of Haskell. As this
language is not very user-friendly, the GNU Guix project
investigated in the usage of the Scheme language as an
alternative to Nix
\cite{courtesFunctionalPackageManagement2013} .

A different approach to the problem of mutating packages in
the operating system has been taken by OSTree. This project
aims to develop a package-centric installation model based
on git \cite{waltersFutureContinuousIntegration2013} . In
the OSTree-backed distributions, like Fedora Silverblue, the
operating system is stored in a ostree, such that changes
updates are done by pulling a new ``commit'' from the
internet. This change is done automatically without user
intervention, such that the files in the operating system
are not swapped while the system is online. As ostree
organizes the file in a git-like structure, the \ac{FHS} is
kept (|/usr/bin|, |/usr/lib|, etc), but the files are stored
in a content address storage. The file system is also kept
as read-only, as the changes are performed on boot, thus
having an immutable system.

While package managers for Linux operating systems are on
the spotlight of this project, any software project grows
big enough such that it needs to implement its own package
manager. This is the case for every programming language,
that supports any kind of concept of libraries, such as the
languages used for this project: Rust (crates) and Lua
(luarocks). The reproducibility of packages is of serious
concern \cite{goswamiInvestigatingReproducibilityNPM2020}, as the world relies more heavily on online
services.
