\FloatBarrier
\chapter{Implementation}

The following chapter describes the implementation of miq.
Miq is the program that was developed as a proof of concept
of the topics described in this document. As such, miq
(pronounced [miku]), is a package manager for Linux. It can
be categorized as a ``source-based'' package manager,
because the user would build every package from its sources
(for example, Gentoo's package manager |emerge| or |nix|),
as opposed to to a ``binary-based'' package manager, which
distributes the already-built packages to its users.

The model of a source-based package manager provides some
advantages, such as being able to edit the recipes that
build the packages. Moreover, by integrating the tool that
builds the packages and the tool that installs them, the
user gets a fast feedback loop to iterate over the writing
of the package definitions. In a source-based package
manager, the user receives a copy of a source tree that
contains the recipes, definitions of all the packages in the
repository. These files contain the instructions about all
the packages that are available, and the machine-readable
data for the package manager to build them. The user would
tell the package manager to build or install certain package
X, and the program would then:

\begin{itemize}
    \item Evaluate the file tree of package definitions.
    \item Calculate a transaction (which files to build or download)
    \item Perform the transaction
\end{itemize}


miq is heavily inspired by nix, the functional package
manager, a work of \Citeauthor{dolstraNixOS2008}
\cite{dolstraNixOS2008}. The work of nix of using a
hash-based path system instead of the \ac{FHS} was the idea
that lead this project, and that is implemented in miq by
following the same conventions that in nix. This means, that
packages are configured with a custom prefix that lives in
the \textbf{store} (|/miq/store| and |/nix/store| respectively), with
a unique subdirectory based on the hash of the package
definition -- which implies that any change in its
dependencies would trigger a recalculation of the store
path. Miq differs with nix in that the schema for the
calculation of the store path follows the pattern
|name-version-hash|, while nix uses |hash-name-version|.
While the result of the package is functionally the same, by
using the name first, the list of folders in |/miq/store|
can be easily ordered by package name.

Miq differs from nix in the usage of the language used for
the evaluation of the packages. On of the powerful features
of both miq and nix is the ability to declare packages as
code, using a turing-complete programming language. This is
in contrast to apt or rpm, which use a markup language to
define the packages. By using a turing-complete language,
the user is able to create any kind of abstraction around
the package primitives. This not only allows for code
deduplication, but also to easily create new ``helper''
packages that would otherwise not be considered with a
classical markup-language based package manager.

In this regard, for nix, a new programming language was
invented with the same name, nix. This is a purely
functional language inspired by the semantics of Haskell, and
often referred to as ``JSON with functions''. As this
language is sometimes alien to users, with miq the decision
was to use Lua, an easily emendable language. This is
discussed in more depth in section \ref{sec:lua} .

Miq is implemented in the Rust programming language
\cite{RustProgrammingLanguage} . Rust is a relatively new
language created at Mozilla, which focuses on the ability to
build fast and reliable programs. This is achieved through
the use of rich type system, that guards the programmer from
making mistakes and eases the process of a refactor. Rust
also provides some other features that were viable to this
project. A very important detail is that Rust can build
native binaries (ELF) that can be run by the operating
system. Every dependency of the Rust program (crates) are
also built into the final binary, and C library requirements
(libc) are statically linked. This means that the final
result is a single |20MB| binary, with no dependencies that can be
run in any Linux system, no matter the distribution.

\section{Units: the basic building blocks}

The main purpose of the package manager, is to build
packages according to a dependency tree. Therefore, it is
important to properly define what a ``package'' is and is
not. In miq, the concept of a package has been abstracted by
the concept of a ``unit''. A unit is something that can be
built, and put into the miq store (|/miq/store|). This
abstraction has been chosen, because it makes it easy to
have different entities that can be built. This
differentiation serves the purpose of being able to declare
2 types of units: package units, and fetchable units.
As discussed in section \ref{sec:builder}, package units are
built in a network sandbox, to avoid impurities resulting of
a connection to the network. The main aspect of miq is being
able to hash packages depending on its dependencies and
build script. The ``result'' of building the package is not
considered in the hashing algorithm. This can be described
as packages being input-addressed, instead of
content-addressed. As such, it is important to have a proper
build sandbox to isolate the build process and produce a
reproducible result, for the same input parameters. The
purpose of fetchable units (|Fetch|) is to be able to split
the network connection from the build process. The
implementation for |Unit| is a enumeration with 2 cases:

\begin{minted}{rust}
#[derive(Clone, Debug, Serialize, Deserialize, Hash)]
enum Unit {
    PackageUnit(Package),
    FetchUnit(Fetch),
}
\end{minted}

This example already shows one of the features of Rust that
made the development of miq ergonomic: |derive| macros.
|derive| is a keyword that allows for automatic code
generation given some |struct| or |enum|. In this case the
|Hash| trait is derived. What this means is that Rust
automatically generates an implementation for the |Hash|
trait, given that the fields of the enum are also |Hash|.

Very importantly too, the traits |Serialize| and
|Deserialize| are derived for Unit. These traits are
provided by the |serde| crate (serialization and
deserialization) \cite{Serde} . Similarly to the |Hash|
derive macro, serde allows to automatically generate
serialize and deserialize a Unit from any crate that
provides an interface for it. This comes into play for the
usage of the intermediate representation objects that are
store into disk, in |toml| format. As discussed in the next
section, miq loads the units from disk in |toml| format, and
automatically parses them into the data structure, without
any need of parsing code.

The Unit enum has two members, which wrap the inner values:
|Package| and |Fetch|. These are structs that contain all
the information that miq uses to build the Unit. The
implementation of the Fetch unit is trivial, as the
following code snippet shows:

\begin{minted}{rust}
#[derive(Deserialize, Serialize, Hash)]
struct Fetch {
    result: MiqResult,
    name: String,
    url: String,
    executable: bool,
}
\end{minted}

The toml format is an human-readable serialization format,
that can be compared to JSON. As the latter is not
considered ergonomic to manually write, it was decided to
use the toml format instead, although this doesn't have any
impact on the functioning of the package manager. The
following snippet of code shows a serialization of the musl
Fetch Unit.

\begin{minted}{toml}
# /miq/eval/musl-1.2.3.tar.gz-828bf8f78328fb26.toml
type = "FetchUnit"
result = "musl-1.2.3.tar.gz-828bf8f78328fb26"
name = "musl-1.2.3.tar.gz"
url = "https://musl.libc.org/releases/musl-1.2.3.tar.gz"
executable = false
\end{minted}

The example above already shows the hashing of the Unit
itself: the |result| field. This field is computed as a
function of the |url| and the |executable| fields, and
determines the hashing of the Unit itself. This is stored
attached to the Unit itself. This example is trivial, but
any other Fetch unit with the same |name|
(|musl-1.2.3.tar.gz|) but different |url|, would trigger a
change in the hashing that ends up in |result|, as shown in
figure \ref{fig:fetch_hash} .

\begin{figure}[hbt]
    \centerfloat
    \includesvg[width=350pt]{assets/fetch_hash.svg}
    \caption{Hashing process of a Fetch Unit.}
    \label{fig:fetch_hash}
\end{figure}

Similarly, a Package Unit is written as a struct with the
instructions required to build it. The following simplified
snippet shows the implementation in Rust:

\begin{minted}{rust}
#[derive(Serialize, Deserialize, Hash)]
pub struct Package {
    result: MiqResult,
    name: String,
    version: Option<String>,
    deps: BTreeSet<MiqResult>,
    script: String,
    env: BTreeMap<String, String>,
}
\end{minted}

An example serialization of a Package Unit for musl is the
following:

\begin{minted}{toml}
# /miq/eval/musl-f0dd14ee1ca91c64.toml
type = "PackageUnit"
result = "musl-f0dd14ee1ca91c64"
name = "musl"
version = "1.2.3"
deps = [
    "musl-1.2.3.tar.gz-unpack-5f9d5116c4c83592",
    "stage0-stdenv-d2ecc89c54b1b316",
]
script = '''
source /miq/store/stage0-stdenv-d2ecc89c54b1b316/stdenv.sh
set -ex
/miq/store/musl-1.2.3.tar.gz-unpack-5f9d5116c4c83592/configure \
    --prefix=$PREFIX \
    --disable-static \
    --enable-wrapper=all \
    --syslibdir=$PREFIX/lib
make -j$(nproc)
make -j$(nproc) install
ln -vs $PREFIX/lib/libc.so $PREFIX/bin/ldd
'''
\end{minted}

A Package Unit is more complex than a Fetch Unit, because of
the following:
\begin{itemize}
    \item Package units have a |deps| field. |deps| are the
    |result| strings of any other Unit. This means Package
    Units can depend on other Package or Fetch Units, while
    Fetch Units don't depend on anything.
    \item The |build| field is a literal bash script that is
    used during the build process to produce the output.
\end{itemize}

In contrast to other package managers and similarly to nix,
miq doesn't provide multiple stages for the build process.
In Gentoo, the definition of package, usually contains some
unpack, build and install phases. In contrast, everything in
miq is done through the script section. This is because, by
using a language to generate this |toml| representation, the
user is free to include any custom functionality as needed.
This is important, as the Package Unit tries to be as
minimal as possible, by just providing the |script| phase,
and letting the user construct any abstraction on top of it.
In the example, musl depends on |stdenv|, which is a script
that sets some environment variables required for the build
process, effectively implementing the concept of build
stages in other package managers.

\begin{figure}[hbt]
    \centerfloat
    \includesvg[width=350pt]{assets/phash.svg}
    \caption{Hashing process of a Package Unit.}
    \label{fig:pkg_hash}
\end{figure}

The hashing algorithm used for the process is the
\textit{Fowler Noll Vo} hash function \cite{FnvRust}. While
this algorithm is not cryptographically secure, and doesn't
provide protection against collision attacks, the algorithm
could be considered as sufficient for this application. This
could be considered as an implementation detail, and could
be swapped for any algorithm such as |SHA256| if deemed
necessary.

The implementation of miq uses a 2-stage evaluator, that
first converts the Lua code into a serialization of Units,
and then serializes them again to sort the dependency
algorithm. Unit files are stored in the |/miq/eval|
directory, generated and consumed by miq in sequence, as
shown in figure \ref{fig:2stage} .

\begin{figure}[hbt]
    \centerfloat
    \includesvg[width=400pt]{assets/22stage.svg}\
    \caption{Two-staged evaluation process.}
    \label{fig:2stage}
\end{figure}

The purpose of this 2-stage evaluator process, is be able to
extend miq as needed. Because the |unit.toml| format is
well-known, a user or contributor should be able to
implement a different evaluator in any other language than
Lua. This is one of the main shortcomings of nix, as it
only allows the usage of the nix language. The Guix package
manager \cite{courtesFunctionalPackageManagement2013}
studied the usage of Scheme as the language for the
evaluation process, by patching into the nix builder. By
explicitly allowing for an intermediate format, the
ergonomics of the package manager could be improved.

As a side-effect of this design, some characteristics
emerge:

\begin{itemize}
    \item Units toml files can be manually written by hand.
    During the development process of miq, the Lua evaluator
    was the last piece implemented, because Unit files could
    be manually written as a proof of concept.
    \item Units can be copied and pasted from one machine to
    another. If the reproducibility of the evaluator cannot
    be assured, unit files can be checked out into version
    control. A user could then use the locally downloaded
    unit tree to build the resulting packages. This could be
    useful as an alternative distribution method to the Lua files.
\end{itemize}

\section{Dependency solver}

\section{Recipe evaluator}
\label{sec:lua}

\section{Builder}
\label{sec:builder}

\section{Database}
