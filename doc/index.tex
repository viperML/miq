\documentclass[
  a4paper,
  % twoside
]{report}

\title{Immutable package management for Linux}
\author{Fernando Ayats Llamas}

% Better fonts
\usepackage{fontspec}
\setmainfont{texgyrepagella}[
  Extension = .otf,
  UprightFont = *-regular,
  BoldFont = *-bold,
  ItalicFont = *-italic,
  BoldItalicFont = *-bolditalic,
  ]
\setmonofont{iosevka-normal-medium.ttf}[
  SizeFeatures={Size=10},
  Contextuals={Alternate},
  NFSSFamily={iosevka-normal}
]

% Bibliography (no bibtex)
\usepackage[backend=biber]{biblatex}
\addbibresource{assets/miq.bib}

\usepackage{graphicx}
\usepackage{svg}
\svgsetup{inkscapelatex=false}


\usepackage{xurl}


\usepackage[colorlinks]{hyperref}
\usepackage{xcolor}
% \hypersetup{pdfborder = {0 0 0}} % no boxes around links
\definecolor{MK_One_One}{RGB}{140,81,10}
\definecolor{MK_One_Two}{RGB}{216,179,101}
\definecolor{MK_One_Three}{RGB}{246,232,195}
\definecolor{MK_One_Four}{RGB}{199,234,229}
\definecolor{MK_One_Five}{RGB}{90,180,172}
\definecolor{MK_One_Six}{RGB}{1,102,94}
\hypersetup{
 linkcolor=MK_One_One
,citecolor=MK_One_Two
,filecolor=MK_One_Three
,urlcolor= MK_One_Six
,menucolor=MK_One_Five
,runcolor=MK_One_Four
,linkbordercolor=MK_One_One
,citebordercolor=MK_One_Two
,filebordercolor=MK_One_Three
,urlbordercolor=MK_One_Six
,menubordercolor=MK_One_Five
,runbordercolor=MK_One_Four
}


\usepackage{lipsum}

\usepackage{acronym}

\makeatletter
\AtBeginDocument{%
  \renewcommand*{\AC@hyperlink}[2]{%
    \begingroup
      \hypersetup{hidelinks}%
      \hyperlink{#1}{#2}%
    \endgroup
  }%
}
\makeatother

\usepackage{listings}
\lstMakeShortInline[columns=fixed]|
\lstset{
  basicstyle=\ttfamily,
  basewidth=0.5em
}

% custom headings
% \usepackage{fancyhdr}
% \pagestyle{fancy}
% \fancyhf{}
% % \lhead{\rightmark}
% % \rhead{Página \thepage}
% %% \cfoot{\today}
% \fancyhead[LE,RO]{\rightmark}
% \fancyhead[RE,LO]{Page \thepage}
% \fancyfoot[CE,CO]{Fernando Ayats Llamas - \today}

\usepackage[
  outputdir={aux}
]{minted}
% \usemintedstyle{vs}

% https://tex.stackexchange.com/questions/16582/center-figure-that-is-wider-than-textwidth
\makeatletter
\newcommand*{\centerfloat}{%
  \parindent \z@
  \leftskip \z@ \@plus 1fil \@minus \textwidth
  \rightskip\leftskip
  \parfillskip \z@skip}
\makeatother

\usepackage{placeins}

\usepackage{tabularray}

\makeatletter
\newcommand{\unchapter}[1]{%
  \begingroup
  \let\@makechapterhead\@gobble % make \@makechapterhead do nothing
  \chapter*{#1}
  \addcontentsline{toc}{chapter}{#1}
  \endgroup
}
\makeatother

\usepackage{pdfpages}

\providecommand{\keywords}[1]
{
  % \large
  \textbf{Keywords ---}
  \normalsize
  #1
}
\begin{document}

\includepdf[pages=-]{assets/Portada-externa.pdf}
\shipout\null

\includepdf[pages=-]{assets/Primera-interna.pdf}
\includepdf[pages=-]{assets/Segunda-interna.pdf}

\shipout\null

\begin{abstract}
  Package management is an invariant of all Linux distributions, as it is a
necessary piece to building the software that runs on them. For this reason,
optimizing the package management process is a key factor in the correctness and
performance of the resulting system. Traditional Linux distributions, like
\textit{Debian} or \textit{Fedora} --- and their respective
package managers \textit{apt} and \textit{dnf} --- rely on a
model of sharing every dependency as much as possible, the so-called File
Hierarchy Standard. This model of software
deployment traces back to the early days of Unix in the seventies, and has not
received any fundamental changes since then. As software grows in complexity as
time passes, this models starts to show its limitations. The most notable
solution to this, is the containers' deployment model, in which an Operating
System is encapsulated from the host, but looking like a standard OS from the inside.


In this work, a new approach to classical package
management is presented, based on the concepts of immutability and
Directed Acyclic Graphs. This data structure is used to represent the tree of dependencies
that the package manager resolves. With the usage of a
special file system
layout, many problems of the classical software distribution model are solved,
such as the Diamond Dependency Problem or software rollbacks.
The ideas presented on these documents are inspired by the nix package manager,
from which many concepts are borrowed. As a result, the package manager miq was
developed, which implements this new approach.


  \vspace{10pt}

  \begin{center}
    \keywords{Linux, package, immutable, declarative, graph}

    \textbf{Palabras clave ---} Linux, paquetes, inmutable, declarativo, grafo
  \end{center}
\end{abstract}


\newpage

% \renewcommand{\abstractname}{Acknowledgements}
% \begin{abstract}
%   \input{_acknowledgements.tex}
% \end{abstract}
% \newpage


\begin{titlepage}
  \tableofcontents
\end{titlepage}


% c) Un trabajo de carácter investigador que incluya, al menos, los siguientes apartados:
% objetivos, estudio del estado del arte, hipótesis,
% metodología, resultados, discusión, conclusiones y
% bibliografía.

\chapter{Introduction}


To understand the problems of the classical software deployment model, we need
to take a step back into its origins. We can go back to the seventies, when the
Unix system was born. The design of file system became the basic design
component of Unix  \cite{ritchieUNIXSystemEvolution1984} . The idea was to be able to interface
with all the devices available to the operating system, as if they were regular
files. For this reason, the layout of this file system was of great importance.
Every file (which include device nodes) is placed as a child of the root
directory |/| . Then, the first level of directories organizes the overall
structure of the system. As Linux distributions emerged and developed, this
structure was preserved, and became the \ac{FHS}
\cite{FHSLinuxFoundation}.

In the \ac{FHS} deployment model, files are organized by separate packages, and the
collected into common directories, such as |/bin| for executable files, or
|/include| for header files. This model provides some advantages, such as

\begin{itemize}
    \item Common place for every file type, which makes it easier to
        find them.
    \item Packages share their common dependencies, which reduces the
        amount of disk space used.
\end{itemize}

To share a dependency, a package |A| only has to use the absolute path to the
file from package |B|, which is known. For example, if package |A| has a binary
|/bin/A| which depends on the library |B| at |/lib/libB.so|, and |C| also
depends on it, we can draw the dependency graph on figure \ref{fig:graph1}.

\begin{figure}
    \centering
    \includegraphics[width=150pt]{Screenshot 2023-05-29 150312.png}
    \caption{Dependency graph of packages A, B and C.}
    \label{fig:graph1}
\end{figure}

This model then couples the dependency graph of the packages, with the structure
of the file system. This means, that not every dependency graph is possible to
be constructed on disk, because we must follow the rules of the file system. For
example, if package |A| wants to use the path |/bin/A|, and package |A'| wants
to use it too, this dependency graph is not possible to be constructed, as we
have a conflict. This is illustrated on figure \ref{fig:graph2}.

\begin{figure}
    \centering
    \includegraphics[width=250pt]{Screenshot 2023-05-29 153351.png}
    \caption{Dependency conflict of packages A and A' .}
    \label{fig:graph2}
\end{figure}

A more subtle problem is the Circular Dependency Problem \cite{al-mutawaShapeCircularDependencies2014}. This problem stems
from using an imperative way of managing the dependencies, that is, the usage of
a \ac{CLI} to install or remove packages one after the
other. After requesting some operation on the dependency graph, the result could have
loops in it, as illustrated on figure \ref{fig:graph3}. This causes problems in
the ordering of the operations, and many others. Colloquially, this is known referred to as
``dependency hell'' \cite{abateDependencySolvingStill2020}, denoting the difficulty of managing the dependencies of a system.

\begin{figure}
    \centering
    \includegraphics[width=0.2\textwidth]{Screenshot 2023-05-29 154056.png}
    \caption{Circular dependency between A and B.}
    \label{fig:graph3}
\end{figure}

What it is proposed then, is a system where the dependency graph is decoupled
from the file system. This means, that packages don't rely on standard locations
to look for files. Therefore, the \ac{FHS} is not used anymore --- or at least, not
in the same way. The previous work of the Linux distributions
NixOS \cite{dolstraPurelyFunctionalSoftware2006} and Guix System
\cite{courtesFunctionalPackageManagement2013} or Spack, the \ac{HPC} package
manager \cite{gamblinSpackPackageManager2015}, already paved the way into this idea.

In miq (pronounced [miku]), the packages are evaluated into a \ac{DAG}, in which
each node represents a package, and each edge represents a dependency. Then,
each node gets a unique identifier, which depends on its own definition, and the
identifier of its dependencies. This ``identifier'' is represented as a
cryptographic hash. Finally, each package receives a
directory in the file system
unique to its identifier, as shown on figure \ref{fig:graph4}.


\begin{figure}
    \centering
    \includegraphics[width=250pt]{Screenshot 2023-05-29 164148.png}
    \caption{Dependency graph of packages A, A', B and C, by using a unique prefix.}
    \label{fig:graph4}
\end{figure}



This system also provides a way to transverse the dependency graph, and look for
any change in the transitive dependency graph. For example, if package |A|
depends on |C| through |B|, a change on |C| triggers a change of its hash. Which
in turns, triggers a change on |B|, and then on |A|. With this system, the full
dependency graph is known. A common problem in Linux is "my application works on
my machine" \cite{mukherjeeFixingDependencyErrors2021}. Even when the application itself is the same, we still rely on
files provided from the system via the \ac{FHS} and the package manager.

Finally, this enables a deployment model where updates don't require modifying
files in place. Traditionally, if we want to update a package with files
|/bin/A| and |/bin/B|, these files are modified in place. This means, that if
there is any problem during the transaction (like a power outage or a disk
failure), the system is left in an inconsistent state. In miq, the original
package would still be available on its old path, and we can delay the update to
just swapping a symlink into the current version.

The main solution to this problem currently is the usage of containers. A
container is a collection of files, which can be thought as Linux distribution
itself \cite{DockerAcceleratedContainerized2022} \cite{merkelDockerLightweightLinux2014}. This container is then run by a ``runtime'', that isolates it from the
host's file system, as shown in figure \ref{fig:graph5} . As a result, the files in the container don't conflict with
the hosts, or other containers. The main way to deploy applications is then
using one container per application, which bundles its entire dependency tree.
Because the underlying file system of each container has the same problems as any
other, the containers solution moves the problem into a singular image, which
contains a single application, and as a result a smaller dependency footprint.

In contrast to miq, where each dependency lives in its own directory, without
collisions and need for a sandboxing runtime, as illustrated in figure
\ref{fig:graph6} .

\begin{figure}[hbt]
    \centering
    \includegraphics[width=200pt]{Screenshot 2023-05-29 173501.png}
    \caption{Deployment model of containerized applications.}
    \label{fig:graph5}
\end{figure}

\begin{figure}[hbt]
    \centering
    \includegraphics[width=200pt]{dep2.png}
    \caption{Deployment model of miq.}
    \label{fig:graph6}
\end{figure}




\chapter{State of the art}
\lipsum[20]


\chapter{Development}


\subsection{High level overview}

\subsection{Builder}

\subsection{Intermediate-representation evaluator}

\subsection{Lua evaluator}

\subsection{Other components}


\FloatBarrier
\chapter{Results and discussion}

The development of miq ended up with a working prototype
that can be used to build some packages implemented
in the reference packages repository in Lua. The number of
lines of code of the program is 1200 lines of Rust, and
around 1000 lines of Lua, the latter being much less
complex.

The program is built on GitHub actions, a Continuous
Integration service that builds the application on the
cloud, and attaches it to the GitHub release. The source
code for the project is therefore hosted on GitHub at \url{https://github.com/viperML/miq}.

While the program is a research project, it was developed
with the end-user in mind, by trying to provide a good user
experience via the \ac{CLI}. The application is not a simple
script, but tries to provide a good error messages. The
usage of the Rust programming language also has served as a
learning experience for the author. While the syntax and
semantics of the languages are intimating at first glance,
the type system and library ecosystem greatly boost the
development productivity.

The standard packages' repository implemented in Lua tries
to organize them in multiple stages: |stage0| and |stage1|,
with the purpose of bootstrapping a C compiler by building
it from the previous result, in hopes of getting a result as
pure as possible. As C programs and libraries are the basis
of Linux systems, the C compiler is usually the first target
to try to build with a new package manager. While this was
also true for miq, the complexity of GCC didn't make it
possible to have a working GCC build in the time span of
this project. This is not due to technical limitations of
miq itself, but of properly configuring the build to work
with the new style of hashed paths of miq.

The first stage of the bootstrap process uses nix's
bootstrap tools, downloaded directly from their
repositories. These bootstrap tools contain a sufficiently
recent GCC (version 7), a libc implementation (musl) and
core utilities to be able to bootstrap the C compiler, such
as bash, make, and others. By using nix's bootstrap tools
and utility functions implemented in Lua to abstract the
standard environment, it was possible to compile several of
the dependencies of GCC, including:

\begin{itemize}
    \item Nix bootstrap tools.
    \item musl libc
    \item cc and ld wrappers
    \item GMP -- GNU
    Multiple Precision
    Arithmetic Library
    \item MPFR -- GNU Multiple Precision Floating-Point Reliable Library
    \item LIBMPC -- GNU Complex floating-point library.
\end{itemize}

The package with the deepest dependency graph is libmpc (one
of GCC's dependencies). Figure \ref{fig:results} shows its
visual representation. Building libmpc can be done with the
following miq invocation:

\begin{minted}[breaklines]{text}
$ miq build ./pkgs/init.lua#stage1.libmpc

/miq/store/bootstrap-tools.tar.xz-9d678d0fc5041f17
/miq/store/toybox-x86_64-69a4327d80d88104
/miq/store/busybox-33a90b67a497c4d6
/miq/store/m4-1.4.19.tar.bz2-6732a25e4458acb
/miq/store/mpc-1.3.1.tar.gz-cf0aa3bd2a0d6fe0
/miq/store/mpfr-4.2.0.tar.bz2-ea1165b7c0959798
/miq/store/unpack-bootstrap-tools.sh-6949dd1f64cfe7b6
/miq/store/gmp-6.2.1.tar.bz2-a8db6558fa4fba6b
/miq/store/musl-1.2.3.tar.gz-828bf8f78328fb26
/miq/store/bootstrap-5f87f2800c8c639e
/miq/store/gmp-6.2.1.tar.bz2-unpack-33b1305b8e698313
/miq/store/m4-1.4.19.tar.bz2-unpack-2dd3c0568e5cf24a
/miq/store/musl-1.2.3.tar.gz-unpack-5f9d5116c4c83592
/miq/store/cc-wrapper-6666c755718e6aa4
/miq/store/ld-wrapper-668a4212dddee39c
/miq/store/stage0-stdenv-d2ecc89c54b1b316
/miq/store/musl-f0dd14ee1ca91c64
/miq/store/cc-wrapper-a948c296a1a6d88a
/miq/store/ld-wrapper-54bd49b0d1298443
/miq/store/stage0-stdenv-cbfc1da815062410
/miq/store/m4-a132d7d257844060
/miq/store/gmp-4dc253ccbc7d9572
/miq/store/mpc-1.3.1.tar.gz-unpack-2032440f15c6d528
/miq/store/mpfr-4.2.0.tar.bz2-unpack-c0aadc49a8faf78f
/miq/store/mpfr-9a80ac127a402980
/miq/store/libmpc-5d2a3c99a73fb6e6
\end{minted}




\begin{figure}[hbt]
    \centerfloat
    \includesvg[width=500pt]{results.svg}
    \caption{Dependency graph of libmpc.}
    \label{fig:results}
\end{figure}


While the |ld| and |gcc| wrappers performs the modifications needed by
this deployment model -- namely modifying the |RUNPATH|
dynamic section of the resulting |ELF| -- more work is
needed to handle properly the addition of dependencies to
the |RUNPATH| . This should be handled in the Lua code, that
creates the |ld| wrappers. For example, some libraries are
missing from the |RUNPATH| of the libmpc, while libc is
properly handled:

\begin{minted}[breaklines]{text}
$ eu-readelf -d /miq/store/libmpc-5d2a3c99a73fb6e6/lib/libmpc.so | grep -e RUNPATH -e NEEDED
  NEEDED            Shared library: [libmpfr.so.6]
  NEEDED            Shared library: [libgmp.so.10]
  NEEDED            Shared library: [libc.so]
  RUNPATH           Library runpath: [/miq/store/musl-f0dd14ee1ca91c64/lib]
\end{minted}

A simpler program like dash, which is a POSIX shell
implementation, is properly built, as it only requires libc.

\begin{minted}[breaklines]{text}
$ miq build ./pkgs/init.lua#stage1.dash
...
/miq/store/dash-fd40df4f5c3b3d7b

$ eu-readelf -d /miq/store/dash-fd40df4f5c3b3d7b/bin/dash | grep -e RUNPATH -e NEEDED
  NEEDED            Shared library: [libc.so]
  RUNPATH           Library runpath: [/miq/store/musl-f0dd14ee1ca91c64/lib]

$ /miq/store/dash-fd40df4f5c3b3d7b/bin/dash
$ cd /
$ echo *
bin dev efi etc home lib64 miq mnt nix opt proc root run srv sys tmp usr var
\end{minted}

\section{Security}

One of the main advantages of Linux operating systems, is
that the library components that form the \ac{OS} itself can
be easily replaced or updated, compared to Windows where
updates are served by Microsoft in a monolithic fashion.

For this reason, in the classical Linux distribution model,
packages are dynamically linked to each other and the
\acl{FHS} is used. When a security update is pushed, for
example for openssl, all the packages that are linked to its
library don't need to be replaced. As the library that they
depend on resides in a standard location at |/usr/lib|, an
update just replaces the underlying file, without an
application author needing to do anything. Therefore, one
could say that the deployment model of nix and miq poses a
problem, as the packages are "statically" linked to each
other --- but still using ELF dynamic linking. When an
update is pushed to openssl, all the packages that depend on
it, don't get the update, as the ELF files point into the
absolute path of the older openssl at
|/miq/store/older-openssl-hash|.

However, this can be solved by using the environment
variable |LD_LIBRARY_PATH| or |LD_PRELOAD|. As mentioned in the previous
sections, the link-loader has a list of known locations that
it search libraries for, namely:

\begin{itemize}
    \item |LD_PRELOAD| environment variable pointing
    directly to libraries to load even if they are not
    needed.
    \item |LD_LIBRARY_PATH| environment variable pointing to
    a path of libraries.
    \item |RUNPATH| dynamic section of the binary pointing
    to a path of libraries.
    \item The default search paths at |/lib| and |/usr/lib|.
\end{itemize}

While miq embeds the library |RUNPATH|, a system
administrator is still able to override it by setting either
|LD_LIBRARY_PATH| or |LD_PRELOAD| at global scope, like with
|/etc/profile|, such that binaries can ignore the embedded
library path to use a different package.

The deployment model of miq would also make it easy to trace
back the dependencies that a package uses by storing them in
the package database. This would make it easy to find which
packages are affected by a security vulnerability, possible
more so than in the classical Linux distribution model.

And in the world of Docker containers, where applications
are built into a micro-operating system constantly, having
to rebuild the whole system because of a package down in the
dependency graph is not so far-fetched from having to
rebuild an entire docker container.

\section{Other applications}

While the scope of this research project was to explore the
problems of the Linux ecosystem, the developed solution
could be useful to other fields. For example, the builder is
not limited to build C libraries, but could also be used to
build python packages with the properties of the bubblewrap
sandbox. This could allow for more reproducible deployments
and easier dependency management. However, it should be
studied how the build should be sandboxed: should python be
a store path? If so, should every dependency be also built
by miq down to libc?

Another application of the builder is to declare files to be
downloaded and cached. By leveraging the Lua scripting
language, it is trivial to programmatically declare files to
build (for example, every major version from N to N + 3).
Fetch Units downloaded by miq are automatically hashed,
making it easy to keep track of different files in the store.


\chapter{Conclusions}

The development of miq has been a way to
explore the possibilities of package management,
and corner cases of the classical model. The
historical baggage of Linux makes it very difficult
to change its status quo, and all the technology
that has been in development tries to perform
incremental improvements over the decades-old
model. With miq, it is shown that a different
approach is possible, even if it means breaking
compatibility with the current model. On the
other hand, many aspects of miq are left not
explored, but iterations over the idea of representing
packages as a structure of independent hashed
nodes, could lead to advantages in both Linux or
any other package management system.

This document could serve as the basis for any future project
that wants to explore the same ideas as miq. The
development of miq itself turned out to take some work
and planning, but the ideas behind it are simple enough
such that the implementation was straightforward. The
process also served as an exploration of the Rust
programming language, which proved to be a very good fit for
this kind of application. The combination of Rust's type
system, along with its ecosystem of libraries and easy
development experience were key factors in the success of
the implementation.

\section{Future work}

As mentioned previously, miq was developed as proof of
concept, without sacrificing much quality of the
implementation or of the user experience. Still, many fields
were left unexplored, and could lead to more research in the
topic. The following list describes some possible
future work that could be done in the area of miq:

\begin{enumerate}
    \item \textbf{Evaluator language}: Lua is the language
    used to evaluate the packages defined by the user. Lua
    turns the |.lua| input files into an intermediate
    representation in |.toml| which follows a
    specification. The usage of Lua was chosen because of
    how easy is to embed, but the language doesn't provide
    the best ergonomics and error messages. The
    implementation of a custom-made domain specific language
    could be considered if no language fulfills the
    requirements needed for the evaluator, but also some
    newer languages are being developed. Some of these
    easily-embeddable languages that were
    considered for this project were Rhai
    \cite{RhaiEmbeddedScripting}, JavaScript (via
    |deno_core| \cite{DenoCoreCrates2023}) or Dhall \cite{DhallConfigurationLanguage}.

    \item \textbf{Builder language}: the POSIX shell is the
    de facto standard for building packages, as it is used
    to define a list of sequential steps to be performed,
    while interacting with other programs, the file system or
    environment variables. However, any other program
    capable of this task could be used. The usage of bash is
    the norm, because it is often a dependency of the core
    packages (libc, gcc), but a builder language might as
    well be embedded into miq itself, such that no external
    requirement is needed. In Guix, this is done by using
    Scheme as both the language for the builder and
    evaluator, which provides a consistent experience.

    \item \textbf{Package toolchain}: miq served to package
    some very simple applications, so one development are
    would be to package more complex applications -- and
    even package miq itself with miq. It is also of interest
    the usage of different toolchains, such as using glibc
    instead of musl, or the Clang/LLVM toolchain instead of GNU's.

    \item \textbf{Cross compilation}: the current dependency
    model of miq doesn't take into account dependencies
    required at build time or at runtime. By adding support
    for this differentiation, it would be possible to have
    miq compile a package in a system A -- for example
    |x86_64| and have it run in a |aarch64| system.

    \item \textbf{Garbage collection}: as miq is based on
    the design of nix, the same problems that nix has with
    garbage collection appear on miq. The issue is that as
    the user builds packages, there is no mechanism to tell
    how to remove unused paths. The solution used by nix is
    to save in the store database the relationship between
    the packages, such that everything can be removed except
    a root package and all its dependencies.


\end{enumerate}


\newpage
\addcontentsline{toc}{chapter}{Bibliography}
\printbibliography[title=Bibliography]

%\unchapter{List of figures}
\addcontentsline{toc}{chapter}{List of figures}
\listoffigures

\unchapter{List of acronyms}
\begin{acronym}
  \acro{DAG}{Directed Acyclic Graph}
  \acro{FHS}{Filesystem Hierarchy Standard}
  \acro{CLI}{Command Line Interface}
  \acro{HPC}{High-performance computing}
  \acro{VM}{Virtual Machine}
  \acro{OS}{Operating System}
  \acro{PM}{Package Manager}
  \acro{PHT}{Program Header Table}
  \acro{LSB}{Linux Standard Base}
  \acro{CLI}{Command Line Interface}
  \acro{TOFU}{Trust On First Use}
  \acro{PID}{Process ID}
  \acro{ELF}{Executable and Linkable Format}
  \acro{ORM}{Object Relational Mapping}
\end{acronym}


% \chapter{Annexes}
\chapter{Annexes}

\section{Miq installation}

Miq is an open-source project written in Rust. This allows
to easily build the project into a single executable that
bundles all its dependencies. The only C dependencies (libc
and sqlite3) are statically linked too.

By using GitHub actions, the latest version of the project
is compiled and pushed into the ``releases'' section of the
repository. It is only built for |x86_64| Linux machines.

To download the latest version of the project, follow the
following instructions:

\begin{enumerate}
    \item Download the release tarball, containing the
    executable itself, a copy of Bubblewrap and the Lua
    files that describe the packages, optionally changing to
    some new directory:
\begin{minted}{text}
$ mkdir ~/miq && cd ~/miq
$ curl -OL https://github.com/viperML/miq/releases/download/latest/release.tar.gz
\end{minted}

    \item Extract the tarball:
\begin{minted}{text}
tar -xvf release.tar.gz
\end{minted}


    \item Miq is ready to be used
\begin{minted}{text}
$ ./miq --help
miq 0.1.0
Fernando Ayats <ayatsfer@gmail.com>

Usage: miq <COMMAND>

Commands:
    build   Build a package
    eval    Evaluate packages
    lua     Reference implementation of the evaluator, in Lua
    store   Query the file storage
    schema  Generate the IR schema

Options:
    -h, --help     Print help
    -V, --version  Print version
\end{minted}

\end{enumerate}




\section{Miq usage manual}

The following subsections describe the usage of the \ac{CLI}
of miq. While short descriptions can be accessed through the
|--help| flags for every subcommand, this document goes into
more detail about each function.
Miq should have been installed according to the instructions
in the previous section.

As discussed in chapter
\ref{chap:overview}, miq is composed of 4 main components: a
2-stage evaluator, the builder and the database handler,
displayed in figure \ref{fig:miq-components}. These
components can be accessed sequentially, by using the
appropriate subcommands.

\subsection{Evaluating packages}
\label{sec:eval}

The most basic function of miq is to only run the package
evaluator. The packages are described in Lua files, which
are run and produce a dependency graph in memory. The
evaluator can be run separately without building any
package, and performs the two stages:

\begin{enumerate}
    \item Run the Lua code, which produces intermediate
    representation Unit files in |/miq/eval/*.toml| .
    \item Parse the Unit files and calculate the
    dependencies between them to produce a dependency \ac{DAG}.
\end{enumerate}

Step 1 can be skipped, if the user decides to use a
pre-evaluated toml file.

The entrypoint in the \ac{CLI} is the subcommand |miq eval|.

\begin{minted}{text}
$ miq eval --help
Evaluate packages

Usage: miq eval [OPTIONS] <UNIT_REF>

Arguments:
    <UNIT_REF>  Unitref to evaluate

Options:
    -o, --output-file <OUTPUT_FILE>  Write the resulting graph to this file
    -n, --no-dag
        --eval-paths                 Print eval paths instead of names
    -h, --help                       Print help
\end{minted}

|miq eval| takes a Unit reference. Currently there two types
of Unit references:

\begin{itemize}
    \item Serialized Unit reference. This is a direct
    reference to a Unit file |/miq/eval/*.toml|, which skips
    the Lua evaluator.
    \item Lua Unit reference. This reference has two parts
    separated by a hash symbol: |file.lua#item| .
\end{itemize}

Within the miq release tarball, it is distributed a set of
Lua files written for this project that can build some
applications. The main entry point of these files is
|init.lua| . The Lua file pointed by the \ac{CLI} must
return a table, of which it can be selected the key to
build (nested tables are supported, by using the syntax
|file.lua#parent.child|).

After evaluating a Lua file, miq would print any errors
encountered, or otherwise a |dot| language representation of
the \ac{DAG}, as the following example shows:

\begin{minted}{text}
$ miq eval ./init.lua#stage0.bootstrap
digraph {
    0 [ label = "bootstrap" ]
    1 [ label = "bootstrap-tools.tar.xz", shape=box, color=gray70 ]
    2 [ label = "busybox", shape=box, color=gray70 ]
    3 [ label = "toybox-x86_64", shape=box, color=gray70 ]
    4 [ label = "unpack-bootstrap-tools.sh", shape=box, color=gray70 ]
    0 -> 1 [ ]
    0 -> 2 [ ]
    0 -> 3 [ ]
    0 -> 4 [ ]
}
\end{minted}

This graph can be manually inspected to check if the result
matches the desired output. With the |dot| tool from the
|graphviz| package, this graph can be rendered into an
image, as shown by the following example:

\begin{minted}{text}
$ miq eval ./init.lua#stage0.bootstrap | dot -Tsvg > graph.svg
\end{minted}

\begin{figure}[hbt]
    \centerfloat
    \includesvg[width=300pt]{assets/example.svg}
    \caption{Example of a dependency graph produce by miq eval.}
    \label{fig:miq-eval-graph}
\end{figure}

Nodes with a dim boxed outlines are Fetch Units, while round
default nodes represent Package Units.


\FloatBarrier
\subsection{Lua API}

As mentioned in this project, miq uses Lua as a scripting
language to describe packages, instead of using |bash| like
some Linux distributions like Gentoo or Arch Linux, for their
ebuilds or PKGBUILDs respectively.

The Lua runtime is completely embedded into the miq
executable, and allows for a very ergonomic and powerful
interface between the scripting language and the Rust native
code.

The Lua evaluator takes a top-level Lua file, which must
return a Lua table. This table may contain any number of
key-value pairs of package names to its Units, but nested
tables are also allowed, as well as any other data type that
can be ignored. The Rust runtime inserts a library into the
available functions, so no Lua code is required to import
the standard library:

\begin{minted}{lua}
local miq = require("miq")
\end{minted}

The available functions are:

\begin{itemize}
    \item |miq.fetch|: Function that creates a Unit of kind
    Fetch. The input must be a table with the following
    keys:
    \begin{minted}{text}
url = <string> -- URL to fetch
executable = <bool> -- Wether to set the file as executable. Optional, default is false.
    \end{minted}

    \item |miq.package|: Function that creates a Unit of
    Package. The table input must contain the following
    keys:
    \begin{minted}{text}
name = <string> -- Name of the package
version = <string> -- Version of the package. Optional default is empty.
script = <string|MetaText> -- Bash script to execute during
build. May be the output of the f function.
env = <table> -- Environment variables to set during build. Optional.
\end{minted}

    \item |miq.f|: Function that takes a single string as
    input. The string is parsed looking for the characters
    |{{ }}|, and for every match, it substitutes the name of
    the variable, similar to how f-strings work in Python.
    |f| can substitute strings into strings, but also Units
    into strings. When a Unit is substituted into a string,
    a different value is returned: a MetaText. A MetaText is
    just a table containing the string with a list of
    dependencies.
    \begin{itemize}
        \item The string is the result of substituting the
        store path of the Unit .
        \item The list of dependencies gets appended the
        Unit that was substituted.
    \end{itemize}
    |f| also supports interpolating strings or Units into a
    MetaText.

    \item |miq.trace|: Function that takes any Lua value and
    logs it into console, by printing a pretty
    representation. Lua tables are usually printed by just
    the numeric representation of pointer into the heap,
    while |miq.trace| prints the table as a list of
    key-value pairs.
\end{itemize}

While Units are serialized into plain Lua tables that can be
manipulated by hand, it is advised to only use Units that
are produced by the API functions, as the Rust runtime
relies on the exact keys being present in the table.

On the other hand, a set of functions are implemented in
native Lua, that wrap the API functions, as short hands to
easily represent the packages. The functions are organized
into different files, which can be ``required'' by each
other or in the top level |init.lua|.

\begin{minted}{lua}
local utils = require("utils")
local stage0 = require("stage0")
local stage1 = require("stage1")
\end{minted}

Moreover, the list of packages is organized into different
``stages'', with the purpose of trying to build a C compiler
cleanly, by progressively building the tools required to do
so and using them into the next stage. These stages contain
examples of how packages may be implemented.


\subsection{Building packages}

To build a package, the command used is |miq build| :
\begin{minted}[breaklines]{text}
$ miq build --help
Build a package

Usage: miq build [OPTIONS] <UNIT_REF>

Arguments:
  <UNIT_REF>  Unitref to build

Options:
  -q, --quiet            Don't show build output
  -r, --rebuild          Rebuild the selected element, but don't rebuild its dependency tree
  -R, --rebuild-all      Rebuild all packages in the dependency tree
  -j, --jobs <MAX_JOBS>  Maximum number of concurrent build jobs. Fetch jobs are parallelized automatically [default: 1]
  -h, --help             Print help
\end{minted}

As discussed in the previous section \ref{sec:eval}, the
\ac{CLI} uses the concept of Unit references to either use a
raw toml Unit from |/miq/eval|, or to run a Lua script that
evaluates the necessary packages. With the same syntax as
|miq eval|, one can perform a build for a selected package:

\begin{minted}[breaklines]{text}
/miq/store/unpack-bootstrap-tools.sh-6949dd1f64cfe7b6 <- (Fetch { name: "unpack-bootstrap-tools.sh" },)
/miq/store/busybox-33a90b67a497c4d6 <- (Fetch { name: "busybox" },)
/miq/store/toybox-x86_64-69a4327d80d88104 <- (Fetch { name: "toybox-x86_64" },)
/miq/store/bootstrap-tools.tar.xz-9d678d0fc5041f17 <- (Fetch { name: "bootstrap-tools.tar.xz" },)
bootstrap>>+ /miq/store/toybox-x86_64-69a4327d80d88104 mkdir -p /build/bin
bootstrap>>+ export PATH=/build/bin:/no-such-path
bootstrap>>+ PATH=/build/bin:/no-such-path
bootstrap>>+ /miq/store/toybox-x86_64-69a4327d80d88104 ln -vs /miq/store/toybox-x86_64-69a4327d80d88104 /build/bin/ln
bootstrap>>'/build/bin/ln' ->
'/miq/store/toybox-x86_64-69a4327d80d88104'

...
\end{minted}

As builds are performed concurrently, the output shows a
prefix for the package that is outputting the log message,
displayed with |package>>|. After a Unit is built, its store
path is printed into the console, so that the output
contents can be manually inspected.

While Fetch jobs are automatically fetched in parallel,
Package Units are not by default, to be able to inspect the
logging messages properly. The user may want to increase
this limit with the |--jobs| option.

The flags |--rebuild| and |--rebuild-all| can also be used
to rebuild either the package that is selected, or all the
packages in the dependency graph, respectively. This may be
useful is some change in the miq source code is made that
doesn't reflect in a store path change, thus the package
being recognized as already built.

\subsection{Querying the store database}

The store is the directory containing all the built Units at
|/miq/store|. Miq uses a sqlite database to keep track of
what files have been written, such that it avoid having to
recompile every file. While the database can be manually
inspected with the \ac{CLI} for |sqlite3|, miq provides a
shorthand for the most common queries via the |miq store|
command:

\begin{minted}[breaklines]{text}
$ miq store --help
Query the file storage

Usage: miq store <COMMAND>

Commands:
    list     List all paths registered [aliases: ls]
    add      Manually register a path
    is-path  Check if a path is registered
    remove   Manually remove a path [aliases: rm]

Options:
    -h, --help  Print help
\end{minted}

The commands are self-explanatory, and should cover basic
operations to check if the store is registering the packages
properly.

One of the most common commands during development is |miq store rm --all|, which can be used to wipe the entire
storage of miq, forcing a rebuild of all packages. A very
useful command when a change in the source code is made such
that it completely changes how the packages are built.

\subsection{Changing the log level}

The log messages are sent to the console by the |tracing|
Rust crate, which can be adjusted to output different log
level with the |tracing_subscriber| crate. By default, only messages of severity |info| or
higher as sent to the console.

This can be changed by using the |RUST_LOG| environment
variable, which uses a custom syntax to declared what should
be logged. The syntax is the following:

\begin{minted}[breaklines]{text}
target[span{field=value}]=level
\end{minted}

For example, to show all tracing messages from miq:

\begin{minted}[breaklines]{text}
$ RUST_LOG=miq=trace miq ...
\end{minted}

More examples of the syntax are the following:

\begin{minted}[breaklines]{text}
RUST_LOG=[luatrace]=trace # Show all tracing messages from the miq.trace Lua function.
RUST_LOG=miq::eval=trace # Only show tracing messages from
the eval module.
RUST_LOG=trace # Show all tracing messages from the
dependencies of miq.
\end{minted}



\end{document}
