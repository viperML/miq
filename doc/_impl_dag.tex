\FloatBarrier
\section{Dependency solver}

The package dependencies are evaluated by constructing a
\acl{DAG} on memory, by reading the intermediate
representation Unit files in |toml| format. As mentioned in
the previous chapter, a Unit is an abstraction over what a
package is, that stores the instructions to perform the build
or fetch process. The \ac{DAG} is
composed of edges of no weight and nodes that are weighted
with Unit's (after being deserialized). The root unit it
selected by the user via the \ac{CLI}. The process to
construct the graph is the following:

\begin{enumerate}
    \item Start at node $N$ with some Unit weight
    \item Read the node's dependencies into a list of Nodes
    \item For each child node $N'$
    \begin{enumerate}
        \item Search the graph for $N'$, and add it if it is
        not present
        \item Add an edge from the child to the parent, such as $N <- N'$
        \item Recursively perform the algorithm for this
        child node
    \end{enumerate}
\end{enumerate}

This recursive process recursively performed, until one of
the branches reach a Fetch Unit. Fetch Units don't have
dependencies, so the stack returns and unfolds into the
parent task that called it. A visual representation of this
process is illustrated in figure \ref{fig:depbuild} .

\begin{figure}[hbt]
    \centerfloat
    \includesvg[width=300pt]{assets/dep_build.svg}
    \caption{Process to build the dependency graph.}
    \label{fig:depbuild}
\end{figure}

More in detail, the process that walks the children of a node
performs the following computations:

\begin{enumerate}
    \item Read the |deps| field of (Package) Unit node.
    |deps| is implemented as a |Vec<MiqResult>|. |MiqResult|
    is a type-alias for a |String|, such that it can safely
    be handled in the proper scenarios. A |MiqResult|
    contains the portion of the store path |<name>-<hash>| .

    \item The |MiqResult| is converted into a |MiqEvalPath|
    . The evaluation paths points into the file that
    produces the Unit, it can be considered a pointer to the
    Unit. This means it is converted to the shape
    |/miq/eval/<name>-<hash>.toml| .

    \item The toml file contained in the |MiqEvalPath| is
    automatically serialized into a Unit, as explained in
    section \ref{sec:unit} .
\end{enumerate}

The \ac{DAG} is stored in memory by creating a mapping of
Node IDs to Units. To store the edges, it is used a mapping Node IDs to
Node IDs 1-to-1, such that direction is preserved. The Rust
crate ``daggy'' \cite{DaggyRust} is used to provide this
data structure, and the required methods to manipulate the
graph, such as reading or writing nodes and edges. On the
type level, it is
constructed a simple type alias to store the result:

\begin{minted}{rust}
type UnitDag = daggy::Dag<Unit, ()>;
\end{minted}
