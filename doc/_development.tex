\chapter{Development}

The result of this research was the development of |miq|, a package manager and
build system for Linux.

miq is a single-file executable that handles the full lifecycle of the build
process of the packages it manages. This stages include:

\begin{enumerate}
    \item Evaluating the expressions that describe packages
    \item Calculating the dependency graph
    \item Fetching the necessary source code
    \item Performing the described build process
    \item Handling the storage and tracking of the installed packages
\end{enumerate}

Therefore, the following sections will all the components that make up miq, and
their interactions.

\section{High level overview}

The development of miq aimed for a modular design, such that
each component didn't have much coupling with the others.
This allows for easy refactoring of parts of the source
code, while leaving the rest of the system untouched. As
such the components of miq can be layed out as the following
diagram:

\begin{figure}
    \centering
    \includesvg{assets/overview.svg}
    \caption{Components of miq}
    \label{fig:miq-components}
\end{figure}

\subsection{Graph based dependencies}

\subsection{Immutability}

\subsection{Stages}

\subsection{Atomic changes}



\section{Builder}

\section{Graph evaluator}

\section{Lua evaluator}

\section{Other components}

