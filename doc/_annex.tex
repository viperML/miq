% \chapter{Annexes}
\chapter{Annexes}

\section{Miq installation}

Miq is an open-source project written in Rust. This allows
to easily build the project into a single executable that
bundles all its dependencies. The only C dependencies (libc
and sqlite3) are statically linked too.

By using GitHub actions, the latest version of the project
is compiled and pushed into the ``releases'' section of the
repository. It is only built for |x86_64| Linux machines.

To download the latest version of the project, follow the
following instructions:

\begin{enumerate}
    \item Download the release tarball, containing the
    executable itself, a copy of Bubblewrap and the Lua
    files that describe the packages, optionally changing to
    some new directory:
\begin{minted}{text}
$ mkdir ~/miq && cd ~/miq
$ curl -OL https://github.com/viperML/miq/releases/download/latest/release.tar.gz
\end{minted}

    \item Extract the tarball:
\begin{minted}{text}
tar -xvf release.tar.gz
\end{minted}

    \item Add the current directory to |PATH|, such that
    |bwrap| can be found (this can be skipped if the host
    machine has Bubblewrap installed):
\begin{minted}{text}
$ export PATH="$PWD:$PATH"
\end{minted}

    \item Miq is ready to be used
\begin{minted}{text}
$ miq --help
miq 0.1.0
Fernando Ayats <ayatsfer@gmail.com>

Usage: miq <COMMAND>

Commands:
    build   Build a package
    eval    Evaluate packages
    lua     Reference implementation of the evaluator, in Lua
    store   Query the file storage
    schema  Generate the IR schema

Options:
    -h, --help     Print help
    -V, --version  Print version
\end{minted}

\end{enumerate}




\section{Miq usage manual}

The following subsections describe the usage of the \ac{CLI}
of miq. While short descriptions can be accessed through the
|--help| flags for every subcommand, this document goes into
more detail about each function.
Miq should have been installed according to the instructions
in the previous section.

As discussed in chapter
\ref{chap:overview}, miq is composed of 4 main components: a
2-stage evaluator, the builder and the database handler,
displayed in figure \ref{fig:miq-components}. These
components can be accessed sequentially, by using the
appropriate subcommands.

\subsection{Evaluating packages}
\label{sec:eval}

The most basic function of miq is to only run the package
evaluator. The packages are described in Lua files, which
are run and produce a dependency graph in memory. The
evaluator can be run separately without building any
package, and performs the two stages:

\begin{enumerate}
    \item Run the Lua code, which produces intermediate
    representation Unit files in |/miq/eval/*.toml| .
    \item Parse the Unit files and calculate the
    dependencies between them to produce a dependency \ac{DAG}.
\end{enumerate}

Step 1 can be skipped, if the user decides to use a
pre-evaluated toml file.

The entrypoint in the \ac{CLI} is the subcommand |miq eval|.

\begin{minted}{text}
$ miq eval --help
Evaluate packages

Usage: miq eval [OPTIONS] <UNIT_REF>

Arguments:
    <UNIT_REF>  Unitref to evaluate

Options:
    -o, --output-file <OUTPUT_FILE>  Write the resulting graph to this file
    -n, --no-dag
        --eval-paths                 Print eval paths instead of names
    -h, --help                       Print help
\end{minted}

|miq eval| takes a Unit reference. Currently there two types
of Unit references:

\begin{itemize}
    \item Serialized Unit reference. This is a direct
    reference to a Unit file |/miq/eval/*.toml|, which skips
    the Lua evaluator.
    \item Lua Unit reference. This reference has two parts
    separated by a hash symbol: |file.lua#item| .
\end{itemize}

Within the miq release tarball, it is distributed a set of
Lua files written for this project that can build some
applications. The main entry point of these files is
|init.lua| . The Lua file pointed by the \ac{CLI} must
return a table, of which it can be selected the key to
build (nested tables are supported, by using the syntax
|file.lua#parent.child|).

After evaluating a Lua file, miq would print any errors
encountered, or otherwise a |dot| language representation of
the \ac{DAG}, as the following example shows:

\begin{minted}{text}
$ miq eval ./init.lua#stage0.bootstrap
digraph {
    0 [ label = "bootstrap" ]
    1 [ label = "bootstrap-tools.tar.xz", shape=box, color=gray70 ]
    2 [ label = "busybox", shape=box, color=gray70 ]
    3 [ label = "toybox-x86_64", shape=box, color=gray70 ]
    4 [ label = "unpack-bootstrap-tools.sh", shape=box, color=gray70 ]
    0 -> 1 [ ]
    0 -> 2 [ ]
    0 -> 3 [ ]
    0 -> 4 [ ]
}
\end{minted}

This graph can be manually inspected to check if the result
matches the desired output. With the |dot| tool from the
|graphviz| package, this graph can be rendered into an
image, as shown by the following example:

\begin{minted}{text}
$ miq eval ./init.lua#stage0.bootstrap | dot -Tsvg > graph.svg
\end{minted}

\begin{figure}[hbt]
    \centerfloat
    \includesvg[width=300pt]{assets/example.svg}
    \caption{Example of a dependency graph produce by miq eval.}
    \label{fig:miq-eval-graph}
\end{figure}

Nodes with a dim boxed outlines are Fetch Units, while round
default nodes represent Package Units.


\FloatBarrier
\subsection{Lua API}

As mentioned in this project, miq uses Lua as a scripting
language to describe packages, instead of using |bash| like
some Linux distributions like Gentoo or Arch Linux, for their
ebuilds or PKGBUILDs respectively.

The Lua runtime is completely embedded into the miq
executable, and allows for a very ergonomic and powerful
interface between the scripting language and the Rust native
code.

The Lua evaluator takes a top-level Lua file, which must
return a Lua table. This table may contain any number of
key-value pairs of package names to its Units, but nested
tables are also allowed, as well as any other data type that
can be ignored. The Rust runtime inserts a library into the
available functions, so no Lua code is required to import
the standard library:

\begin{minted}{lua}
local miq = require("miq")
\end{minted}

The available functions are:

\begin{itemize}
    \item |miq.fetch|: Function that creates a Unit of kind
    Fetch. The input must be a table with the following
    keys:
    \begin{minted}{text}
url = <string> -- URL to fetch
executable = <bool> -- Wether to set the file as
 executable. Optional, default is false.
    \end{minted}

    \item |miq.package|: Function that creates a Unit of
    Package. The table input must contain the following
    keys:
    \begin{minted}{text}
name = <string> -- Name of the package
version = <string> -- Version of the package. Optional default is empty.
script = <string|MetaText> -- Bash script to execute during
build. May be the output of the f function.
env = <table> -- Environment variables to set during build. Optional.
\end{minted}

    \item |miq.f|: Function that takes a single string as
    input. The string is parsed looking for the characters
    |{{ }}|, and for every match, it substitutes the name of
    the variable, similar to how f-strings work in Python.
    |f| can substitute strings into strings, but also Units
    into strings. When a Unit is substituted into a string,
    a different value is returned: a MetaText. A MetaText is
    just a table containing the string with a list of
    dependencies.
    \begin{itemize}
        \item The string is the result of substituting the
        store path of the Unit .
        \item The list of dependencies gets appended the
        Unit that was substituted.
    \end{itemize}
    |f| also supports interpolating strings or Units into a
    MetaText.

    \item |miq.trace|: Function that takes any Lua value and
    logs it into console, by printing a pretty
    representation. Lua tables are usually printed by just
    the numeric representation of pointer into the heap,
    while |miq.trace| prints the table as a list of
    key-value pairs.
\end{itemize}

While Units are serialized into plain Lua tables that can be
manipulated by hand, it is advised to only use Units that
are produced by the API functions, as the Rust runtime
relies on the exact keys being present in the table.

On the other hand, a set of functions are implemented in
native Lua, that wrap the API functions, as short hands to
easily represent the packages. The functions are organized
into different files, which can be ``required'' by each
other or in the top level |init.lua|.

\begin{minted}{lua}
local utils = require("utils")
local stage0 = require("stage0")
local stage1 = require("stage1")
\end{minted}

Moreover, the list of packages is organized into different
``stages'', with the purpose of trying to build a C compiler
cleanly, by progressively building the tools required to do
so and using them into the next stage. These stages contain
examples of how packages may be implemented.


\subsection{Building packages}

To build a package, the command used is |miq build| :
\begin{minted}[breaklines]{text}
$ miq build --help
Build a package

Usage: miq build [OPTIONS] <UNIT_REF>

Arguments:
  <UNIT_REF>  Unitref to build

Options:
  -q, --quiet            Don't show build output
  -r, --rebuild          Rebuild the selected element, but don't rebuild its dependency tree
  -R, --rebuild-all      Rebuild all packages in the dependency tree
  -j, --jobs <MAX_JOBS>  Maximum number of concurrent build jobs. Fetch jobs are parallelized automatically [default: 1]
  -h, --help             Print help
\end{minted}

As discussed in the previous section \ref{sec:eval}, the
\ac{CLI} uses the concept of Unit references to either use a
raw toml Unit from |/miq/eval|, or to run a Lua script that
evaluates the necessary packages. With the same syntax as
|miq eval|, one can perform a build for a selected package:


\subsection{Querying the store}

\subsection{Changing the log level}



\section{Miq development}
