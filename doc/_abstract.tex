Package management is an invariant of all Linux distributions, as it is a
necessary piece to building the software that runs on them. For this reason,
optimizing the package management process is a key factor in the correctness and
performance of the resulting system. Traditional Linux distributions, like
\textit{Debian} or \textit{Fedora} --- and their respective
package managers \textit{apt} and \textit{dnf} --- rely on a
model of sharing every dependency as much as possible, the so-called File
Hierarchy Standard . This model of software
deployment traces back to the early days of Unix in the seventies, and has not
received any fundamental changes since then. As software grows in complexity as
time passes, this models starts to show its limitations. The most notable
solution to this, is the containers deployment model, in which a Operating
System is encapsulated from the host, but looking like a standard OS from the inside.


In this work, a new approach to classical package
management is presented, based on the concepts of immutability and
Directed Acyclic Graphs. This data structure is used to represent the tree of dependencies
that the package manager resolves. With the usage of a special filesystem
layout, many problems of the classical software distribution model are solved,
such as the Diamond Dependency Problem or software rollbacks.
The ideas presented on this documents are inspired by the nix package manager,
from which many concepts are borrowed. As a result, the package manager miq was
developed, which implements this new approach.
