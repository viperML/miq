\FloatBarrier
\section{Package definitions and Lua evaluator}
\label{sec:lua}

Miq is a source-based package manager, which means that the
user downloads a copy of a source tree, which includes all
the definitions for all the packages. These definitions
contain the instructions for the package manager to build
the package into the user's disk. In contrast, a binary-based
package manager has a list of just the packages available
(usually in the form of a database that is synced -- |apt update|), and the user downloads the already built packages
from the distribution's servers.

Miq's deployment model, similarly to Nix's, is to have each
definition of a package reference the exact paths of its
dependencies. A Unit, the intermediate representation of
package definitions, contains a |deps| field, which lists the
packages by its hashed name, and the |script| field may
contain hard-coded paths to these packages. The following
example shows these hard-coded paths:

\begin{minted}{toml}
# /miq/eval/musl-f0dd14ee1ca91c64.toml
type = "PackageUnit"
result = "musl-f0dd14ee1ca91c64"
name = "musl"
version = "1.2.3"
deps = [
    "musl-1.2.3.tar.gz-unpack-5f9d5116c4c83592",
    "stage0-stdenv-d2ecc89c54b1b316",
]
script = '''
source /miq/store/stage0-stdenv-d2ecc89c54b1b316/stdenv.sh
set -ex
/miq/store/musl-1.2.3.tar.gz-unpack-5f9d5116c4c83592/configure \
    --prefix=$PREFIX \
    --disable-static \
    --enable-wrapper=all \
    --syslibdir=$PREFIX/lib
make -j$(nproc)
make -j$(nproc) install
ln -vs $PREFIX/lib/libc.so $PREFIX/bin/ldd
'''
\end{minted}

The purpose of the package evaluator is to resolve these
paths in a step before the build process. A user does not
use the paths directly, as they are calculated by miq. By
using a scripting language, the user would be able to
declare the dependencies in a programmatic way.

To do so, miq uses the Lua programming language. The
previous Unit definition is compiled from the following Lua
code (simplified):

\begin{minted}{lua}
do
    local version = "1.2.3"
    local src = x.fetchTar {
        url = f "https://musl.libc.org/releases/musl-{{version}}.tar.gz",
    }
    x.libc = x.stdenv {
        name = "musl",
        version = version,
        script = f [[
            {{src}}/configure \
                --prefix=$PREFIX \
                --disable-static \
                --enable-wrapper=all \
                --syslibdir=$PREFIX/lib
            make -j$(nproc)
            make -j$(nproc) install
            ln -vs $miq_out/lib/libc.so $miq_out/bin/ldd
        ]],
    }
end
\end{minted}

By using a scripting language to build the intermediate
representation, the absolute paths used by miq are
abstracted away from the user, while still allowing for this
file system model.

The previous Lua example shows the usage of some functions
built specifically for miq. To begin with, the |f| function
is analogous to f-strings in Python, where one can insert
the string representation of a variable into a string. This
|f| function is exported by the |miq| library, which
executes in Rust, and is automatically inserted into the
runtime of the script executing, by using a |require|:

\begin{minted}{lua}
local miq = require("miq") -- added by the Rust runtime
local f = miq.f

local a = 1
local b = f("a is {{a}}") -- => "a is 1"
local c = f "a is {{a}}" -- => "a is 1" (different function calling syntax)
\end{minted}

|f| is implemented using Lua's debugging facilities, which
comes with its drawbacks. The main issue is no editor knows
about the custom syntax for a string (using the |{{}}|
pattern), which means that the editor is not able to warn
the user about a syntax error.

More importantly, the |f| function not only is able to
interpolate strings into strings, as shown in the previous
snippet, but it is also able to interpolate Units. To create
a Unit, a user is able to use the |miq| library to create a
unit. By providing a user input, the fields are hashed and
converted into a Unit, which then is serialized into a Lua
table. The table is the only data structure in Lua, and
holds a mapping of keys to values.

\begin{minted}[obeytabs=true,tabsize=2]{lua}
local miq = require("miq")

local toybox = miq.fetch {
	url = "http://landley.net/toybox/bin/toybox-x86_64",
	executable = true,
}
--[[ => {
  executable = true,
  name = "toybox-x86_64",
  result = "toybox-x86_64-69a4327d80d88104",
  type = "FetchUnit",
  url = "http://landley.net/toybox/bin/toybox-x86_64"
}
--]]
\end{minted}

When |f| is used to interpolate a Unit, instead of returning
a new string, it returns a custom type called |MetaText|. A
|MetaText| is a wrapping type around a string, that also
holds the packages that depend on this text, with the
following Rust type signature:

\begin{minted}{rust}
struct MetaText {
    value: String,
    deps: Vec<MiqResult>,
}
\end{minted}

By interpolating a Unit into a string, the resulting
|MetaText| is used to carry the packages that ``depend'' on
this text. Instead of the user having to manually declare
what dependencies are needed for the package, they can just
directly interpolate the package into a MetaText, and the
information about the dependency is carried over. The result
of interpolating a Unit into a MetaText, is the store path
of the unit (|/miq/store/name-hash|), as shown in the
following Lua snippet:

\begin{minted}{lua}
local miq = require("miq")
local f = miq.f
toybox = miq.fetch {
    url = "http://landley.net/toybox/bin/toybox-x86_64",
    executable = true,
}

local t = f "ls -la {{toybox}}"
--[[ => {
  value = "ls -la /miq/store/toybox-x86_64-69a4327d80d88104",
  deps = { "toybox-x86_64-69a4327d80d88104" }
}
--]]
\end{minted}

Putting everything together, the |f| functions allows the
user to:

\begin{itemize}
    \item Declaratively define the relationships between packages.
    \item Abstract away the absolute paths used by miq, in
    the context of shell scripts.
    \item Automatically append the interpolated packages
    into the dependencies of a package.
\end{itemize}

The usage of Lua as a programming language for scripting the
package definitions comes down to the fact that the language
can be completely embedded into the Rust application. By
using the |mlua| crate \cite{MluaRust} . This library allows
to embed a complete Lua runtime inside a Rust application,
and provides a safe interface to interact with it. Instead
of using any other scripting language like Python, and
relying on message passing and subprocess execution, the Lua
runtime is able to directly communicate with the Rust code
by sending and receiving data. The Rust code is able to call
Lua functions, and the Lua code is similarly able to call
Rust functions easily. While Lua is a dynamically typed
language , the |mlua| crate provides type safety to the
received values, and is able to serialize and deserialize
Rust types into Lua tables. This seamless integration
between the two languages allows for an application written
in an ergonomic language for a big project, while letting
the user the flexibility of a scripting language for the
declaration of the packages.

As a result of using a scripting language to declare the
packages, the writer of the package tree is able to abstract
away common components into smaller functions. Instead of
having to write some boilerplate code around the primitives
(|miq.f|, |miq.package| and |miq.fetch|), one is able to
write ``wrappers'' around the built-in functions. This
allows for a more ergonomic experience around the
primitives, and extension of the functionality.

For example,
the primitive |miq.fetch| creates a Fetch Unit from its
input. A Fetch Unit is simply fetched from the internet, and
stored into the file system. Very commonly, this Fetch is a
tarball that the package script unpacks to build it. So, a
wrapper around |miq.fetch| can be created, such that it
creates an intermediate package that unpacks the tarball.
To implement this in Lua:

\begin{minted}[obeytabs=true,tabsize=2]{lua}
fetchTarBuilder = function(input)
    local input = input

    local fn_result = function(args)
        local args = args
        local input = input
        local post
        if args.post ~= nil then
            post = args.post
        else
            post = "# No post unpack"
        end
        local fetch = miq.fetch(args)
        local pkg = miq.package {
            name = f "{{fetch.name}}-unpack",
            script = f [[
                set -ex
                export PATH="{{input.PATH}}"
                cd $miq_out
                tar -xvf {{fetch}} \
                    --strip-components=1 \
                    --no-same-permissions \
                    --no-same-owner

                {{post}}
            ]],
        }
        return pkg
    end
    return fn_result
end

fetchTar = fetchTarBuilder {
    PATH = f "{{bootstrap}}/bin",
}

fetchTar {
    url = f "https://musl.libc.org/releases/musl-{{version}}.tar.gz",
}
\end{minted}

As can be seen, the |fetchTar| uses some base package that
provides the |tar| executable (in this case from the
|bootstrap|) package to unpack the input url, by using a
Fetch Unit, and a Package Unit that depends on it. While Lua
is not the most powerful language, it allows for some
abstractions over the primitives.

Finally, it should be noted how miq knows what file to
evaluate in the first place. The user is able to provide a
path to either a Unit file (|/miq/eval/name-hash.toml|) or
to a Lua file. If the file extension is |.toml|, then the
Unit is directly read, skipping the evaluator phase directly
into the intermediate representation and dependency
evaluator. Otherwise, the file is take as the ``top-level''
Lua file. The top-level file should return a table, and the
user is able to select which item to build from it. For
example, if the user select the reference
|./pkgs/init.lua#foo|, then |init.lua| is completely
evaluated, and the |foo| key is selected from the resulting
table. This method of references to files with different
extensions could be extended to any other scripting language
used to evaluate the packages, with minimal changes to the
\ac{CLI} parser.
