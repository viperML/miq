\chapter{Conclusions}

The development of miq has been a way to
explore the possibilities of package management,
and corner cases of the classical model. The
historical baggage of Linux makes it very difficult
to change its status quo, and all the technology
that has been in development tries to perform
incremental improvements over the decades-old
model. With miq, it is shown that a different
approach is possible, even if it means breaking
compatibility with the current model. On the
other hand, many aspects of miq are left not
explored, but iterations over the idea of representing
packages as a structure of independent hashed
nodes, could lead to advantages in both Linux or
any other package management system.

This document could serve as the basis for any future project
that wants to explore the same ideas as miq. The
development of miq itself turned out to take some work
and planification, but the ideas behind it are simple enough
such that the implementation was straightforward. The
process also served as an exploration of the Rust
programming language, which proved to be a very good fit for
this kind of application. The combination of Rust's type
system, along with its ecosystem of libraries and easy
development experience were key factors in the success of
the implementation.

\section{Future work}

As mentioned previously, miq is was developed as proof of
concept, without sacrificing much quality of the
implementation or of the user experience. Still, many fields
were left unexplored, and could lead to more research in the
topic. The following list describes some of the possible
future work that could be done in the area of miq:

\begin{enumerate}
    \item \textbf{Evaluator language}: Lua is the language
    used to evaluate the packages defined by the user. Lua
    turns the |.lua| input files into an intermediate
    representation in |.toml| which follows an
    specification. The usage of Lua was chosen because of
    how easy is to embed, but the language doesn't provide
    the best ergonomics and error messages. The
    implementation of a custom-made domain specific language
    could be considered if no language fulfills the
    requirements needed for the evaluator, but also some
    newer languages are being developed. Some of these
    easily-embeddable languages that were
    considered for this project were Rhai
    \cite{RhaiEmbeddedScripting}, JavaScript (via
    |deno_core| \cite{DenoCoreCrates2023}) or Dhall \cite{DhallConfigurationLanguage}.

    \item \textbf{Builder language}: the POSIX shell is the
    de-facto standard for building packages, as it is used
    to define a list of sequential steps to be perfomed,
    while interacting with other programs, the filesystem or
    environment variables. However, any other program
    capable of this task could be used. The usage of bash is
    the norm, because it is often a dependency of the core
    packages (libc, gcc), but a builder language might as
    well be embedded into miq itself, such that no external
    requirement is needed. In Guix, this is done by using
    Scheme as both the language for the builder and
    evaluator, which provides a consistent experience.

    \item \textbf{Package toolchain}: miq served to package
    some very simple applications, so one development are
    would be to package more complex applications -- and
    even package miq itself with miq. It is also of interest
    the usage of different toolchains, such as using glibc
    instead of musl, or the Clang/LLVM toolchain instead of GNU's.

    \item \textbf{Cross compilation}: the current dependency
    model of miq doesn't take into account dependencies
    required at build time or at runtime. By adding support
    for this differentiation, it would be possible to have
    miq compile a package in a system A -- for example
    |x86_64| and have it run in a |aarch64| system.


\end{enumerate}
