\FloatBarrier
\section{Base Linux packages}

To understand the scope of a Linux package manager, we need
to define what constitutes a Linux operating system. To
answer this question, we can take a look at the past, on the
inception on Linux. We can trace back to the UNIX system
\cite{ritchieUNIXSystemEvolution1984} developed by Ken
Thompson, Dennis Ritchie, and others. This operating system
marked a milestone in the history of computing, and the work
of its authors could be divided into 2 research fields: the
development of the interfaces of the OS, and a new
programming language to accompany it, the C programming
language. The UNIX kernel was rewritten in this new
language, and in 1978, the book
``The C Programming Language''
\cite{ritchieProgrammingLanguage1983} was
released, setting the foundation of the language into the
future. As UNIX started to gain popularity, the project
increasingly started to lock users in. As an alternative to
UNIX, the GNU (GNU is Not Unix) project started with the
intention to provide a UNIX-compatible operating system that
would not lock users in with proprietary software licenses.
Finally, the Linux kernel was created by Linus Torvalds
around 1991 as a research project to provide a substitute to
the MINIX kernel, and finally the GNU components replaced
any leftovers of MINIX \cite{OverviewGNUSystem}.

The Linux kernel and all the GNU components were written
in C, so it became the de facto language for system
software. Therefore, the C language is intertwined with how
a Linux operating system works, and all the packages that
compose it from the kernel to userland. From the packager's
perspective, there is one package that is the most
important: \textbf{libc}. The C standard library (libc, or
GNU's libc implementation, glibc) is a collection of C
headers and library objects that provide interfaces to the
operating system for programs written in C. Libc was
standardized in 1999 by the ISO C committee under the C99
language specification, and further revisions were made to
update it. Some headers from libc include:

\begin{figure}[hbt]
    \centerfloat
    \begin{tblr}{hlines, vlines}

        <string.h> & String manipulation \\

        <stdio.h> & Standard input/output \\

        <stdint.h> & Standard integer types \\

        ... & ... \\

    \end{tblr}
    \caption{Some headers from libc.}
\end{figure}

For Linux, there are various implementations of libc, such
GNU's glibc or musl libc \cite{MuslLibc}, being the former
the predominant on all popular Linux distributions. Along
with the headers required by the C ISO standard, the libc
providers are implemented over the kernel's system calls and
also provide UNIX-specific interfaces, namely |<unistd.h>|.
Libc is also the provider of the link-loader discussed in
previous sections (ld-linux.so), and also injects some
loading code in every ELF file, before the main function
execution takes over.

Apart from libc, many programs and libraries are implemented
on top of it, to provide what is called as the \acl{LSB}
\cite{LinuxStandardBase}. This specification outlines some
of the most important programs that compose a Linux system,
such as ``coreutils'', which contains the most basic
terminal utilities, like |ls|, |cat|, |cp|, etc. To be able
to build the components, it is also needed a C compiler,
which is also provided by the GNU project in form of GCC
(GNU Compiler Collection). To
be able to support the compilation and configuration of
programs, some other tools are required like |make| or
|sed|, and also a POSIX shell to interact with the system,
in the form of |bash|. The table \ref{tab:lsb} outlines some
of these components and their functionality.

\begin{figure}[hbt]
    \centerfloat
    \begin{tblr}{hlines, vlines}
        libc & C standard library (glibc, musl) \\

        gcc & C and C++ compiler \\
        coreutils & Basic terminal utilities, like ls,
        cat, cp, etc. \\
        binutils & Binary utilities, like ld, as,
        objdump, etc. \\
         make & Build automation tool \\
         sed & Stream editor, text modification
        program \\
         bash & POSIX shell implementation \\
         tar & Archive utility \\
         grep & Text search utility \\
         findutils & File search utilities, like
        find or findmnt \\
         diffutils & File comparison utilities, like
        diff \\
         systemd & System and service manager \\
    \end{tblr}
    \caption{Basic components of a Linux system.}
    \label{tab:lsb}
\end{figure}

As a Linux distribution grows in scope, more and more
components are needed to build something that an end user is
able to use. Starting from graphical toolkits, like GTK or
QT, to the X11 display server and desktop environments, like
Gnome or KDE Plasma. And every component in-between, like
proper service management via systemd, the package manager
itself, network management, power management, peripherals
support, and all the libraries that are required to build
all the ``leaf'' packages of the full dependency tree.
Finally, the kernel itself is also a package with special
properties (as it doesn't depend on libc), but regardless of
importance.

The scope of this work has been limited to a basic terminal
usage of Linux packages, so the focus won't be on building
complex software, but rather building basic components to
prove the viability of the deployment model. It is also
important to note, that while packages can be put together
to form a fully functional operating system, a
package manager of Linux packages can work as a ``guest'' on
a different distribution. This is, running Linux packages
that are not part of the distribution itself. For a
classical system, this has never been common practice -- for
example, installing apt (Debian's package manager) on a
Gentoo system -- mainly limited by the rules imposed by the
\acl{FHS}. Basically, two package managers would interfere
with each other, as they would try to manage files that were
managed by the other package manager (any file on /usr/bin,
/usr/lib, etc). However, with the usage of the special
hash-based paths, this problem not an issue, as miq can be
used on top of any Linux distribution, without interfering
with it.

\FloatBarrier
\section{The bootstrapping problem}

As discussed in the previous section, the scope of this work
is to be able to demonstrate the viability of the deployment
model of hashed paths instead of relying on the \ac{FHS}.
So, as one of the core libraries, libc is the one to be
compiled the first. As mentioned previously, libc is the C
standard library, that in practice is used by every program
written in C, which make up the core programs of a Linux
operating system. To be able to compile libc, it is required
a C compiler (from GCC), a POSIX shell (bash, for
example) and other utilities. In this project, musl libc was used instead of
glibc, as GNU's version contains some extra additions that
are not part of the standard, which could introduce
complexity. Then, a dependency graph for musl libc can be
drawn as shown in figure \ref{fig:libc-deps}.

\begin{figure}[hbt]
    \centerfloat
    \includesvg[width=150pt]{graph/libc-bs1.dot.svg}
    \caption{Dependency graph for musl libc on first iteration.}
    \label{fig:libc-deps}
\end{figure}

These GCC dependency in this graph is assumed to already
"exist" in the system. But as already discussed, this is a
source of impurity, as we don't know what C compiler was
used, we can't hash the input to produce a hash for the musl
output. This C compiler is ``external'' to the system.
Therefore, one of the objectives of any package manager is to
be able to ``support'' itself, to be able to produce
packages without any external interference. Then, this GCC
must come from the package manager itself as well, and we
can draw its dependency graph in figure \ref{fig:libc-deps2}.

\begin{figure}[hbt]
    \centerfloat
    \includesvg[width=200pt]{graph/libc-bs2.dot.svg}
    \caption{Dependency graph for musl libc on second iteration, recursive nature shows.}
    \label{fig:libc-deps2}
\end{figure}

As can be seen, to be able to build libc, we need a C
compiler, and to build a C compiler we need libc. This is a
problem of a circular dependency, in particular the
bootstrapping problem, because we need to ``base'' to be
able to lift the entire system from the ground.
Moreover, if we try to draw a dependency from cyclically as
in figure \ref{fig:libc-deps3}, we can see that graph is now
\emph{cyclic}, this means that it is no longer a \ac{DAG} as
discussed in the previous section, and we can no longer hash
the nodes, as it would result in an infinite recursion. For
this reason, cyclic dependencies are not allowed in miq, and
using and checking for a proper \ac{DAG} is important.

\begin{figure}[hbt]
    \centerfloat
    \includesvg[width=150pt]{graph/libc-bs3.dot.svg}
    \caption{Cyclic dependency graph for musl libc and gcc.}
    \label{fig:libc-deps3}
\end{figure}

To solve the bootstrapping problem, a solution is proposed
of providing a set of bootstrap tools. This set of tools is
downloaded from the internet, ready to use. By using it as a
``fixed hash'' package, we can break the chain of recursive
dependencies, as shown in figure \ref{fig:libc-deps4}. This
set of bootstrap tools doesn't need to be updated ever. For
example, they could contain a very old GCC, such that it is
recent enough to be able to compile a newer GCC used for the
rest of the packages on the system.

\begin{figure}[hbt]
    \centerfloat
    \includesvg[width=150pt]{graph/libc-bs4.dot.svg}
    \caption{Dependency graph for musl libc with a loop on gcc, solved with a bootstrap package.}
    \label{fig:libc-deps4}
\end{figure}

In practice, the Linux distributions don't use the first
iteration of this process. Instead, an iterative process is
created, where libc and the C compiler (along any extra
tools) are compiled in succession, using the result of the
previous step to compile the next one. These are called
``stages'', and the process ``distills'' the programs until
a stable result is obtained. The final stage (from 0 to 1,
2 , etc) is then used to build the system, instead of the
zeroth stage.

\begin{figure}[hbt]
    \centerfloat
    \includesvg[width=200pt]{graph/libc-bs5.dot.svg}
    \caption{Bootstrap tower with 3 stages.}
    \label{fig:libc-deps5}
\end{figure}
