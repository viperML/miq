\FloatBarrier
\subsection{Immutability}

To build reliable system deployments, it is important to reduce the
number of variables that can affect the outcome. From the
software perspective, this means that the system should
either be reliant to changes in the environment, or minimize
the factors from the host that can affect the application.
In Linux, one of the main factors that can affect the
environment are the libraries that are installed on the
system.

On the previous section it was discussed how a different
approach to tagging the packages on the filesystem can be
used to achieve a consistent environment. What this
filesystem layout naturally leads into, is a system where
there is no mutation of the existing packages. On a
classical system, to upgrade a package, the following steps
are taken:

\begin{enumerate}
    \item Download the new update for package |foo|, and unpack it
    \item Replace file |/usr/bin/foo| with the new version
    \item Replace file |/usr/share/foo-bar| with the new version
    \item \ldots
    \item Register the new version in the database
\end{enumerate}

As can be seen, the process of upgrading a package involves
multiple in-place modifications of the existing package.
This operation can be qualified as ``surgical'', as it may
involve many operations which can fail -- and always
eventually fail. This is discussed on section
\ref{sec:atomicity} .

This modification of the global environment poses a problem
to the running processes on the system. Giving names to the
example packages, lets say that |openssl| is updated to a
new version, which fixes some vulnerabilities. The update
process on a classical system would involve replacing
|libcrypto.so| \textbf{in-place} with the new version. But
any running process -- unless it has some internal mechanism
to detect this change -- will be unaware of this change.
Let's say that a package that depends on |openssl| is
|nginx|, which is linked against |libcrypto.so|. Then the
system may be running a vulnerable version of |nginx|, even
if the package was updated. From the perspective of the
package itself, it hasn't changed, yet the underlying
dependency graph as been altered. A solution to this, would
be to track |openssl|'s ``reverse dependencies'', that is,
all packages that depend on it. With this list of reverse
dependencies, one could think that all you need is to
iterate through it, killing every process that depends on
the dependency. But this is not a trivial solution, as there
is no direct connection of running processes to package
versions. For example, the |nginx| package could declare
some systemd service that is part of the package, and then
restart this service in particular. But the system
administrator could as well have written a custom systemd
service, that is not tracked by the package manager. In the
end, the ``safest'' solution, is to just reboot the entire
system after an update has been applied.

\begin{figure}[htb]
    \centerfloat
    \includesvg[width=250pt]{assets/nginx_classic.svg}
    \caption{Tracking reverse dependencies on a classical Linux distribution.}
    \label{fig:nginx_classic}
\end{figure}


So, if in a classical Linux distribution, tracking the files
of the reverse dependencies is not a reliable way to know
when to restart any service that depends on a mutated file,
then the following step could be made: tracking process
uniquely, with some hashing mechanism, such that it can be
known if the reverse dependency (|nginx|) loaded the
vulnerable dependency (|openssl|).


What this naturally leans into, is the solution proposed in this project, of tracking
every package by hashing its entire dependency tree. This
simple change allows to know whether an affected |nginx| is
running against a vulnerable |openssl|, because the
dependency tree is statically known. Because we don't allow
for mutability, every process always runs linked against the
same exact versions of each library. Therefore, if
|libcrypto.so| is updated, some |nginx| processes will still
be linked to the exact path of the old |libcrypto.so| --
|/miq/store/openssl-version-HASH/lib/libcrypto.so|. By
re-evaluating the dependency graph, we can know that the
hash of |nginx| has changed, because it now depends on a
fixed version. Then, from an administrator's perspective,
all you need yo know is to compare the hashes of the old and
new nginx versions, and restart the service such that it
points into the newly-built version.

\begin{figure}[hbtp]
    \centerfloat
    \includesvg[width=250pt]{assets/nginx_miq.svg}
    \caption{Tracking reverse dependencies by hashing the package paths.}
    \label{fig:nginx_miq}
\end{figure}

\FloatBarrier
\subsection{Atomic transactions}
\label{sec:atomicity}

As discussed in the previous section, the usage on mutable
(in-place modifications) systems on Linux, is a serious
source of problems, because of processing tracking a fixed
path of a dependency, that changes under the hood with a
system update. But a different problem that arises from
mutation is the failure of a transaction, that is that a
transaction is ``not atomic''.

Atomicity in software development is a concept usually
associated with memory management. In a multi-threaded
environment, it is important that shared memory is not
written to by two threads at the same time, but
sequentially. For this purpose, the concept of an ``atomic''
operation means that the operation is either not started, or
completed -- there is no state ``in between''
\cite{neelakantamHardwareAtomicityReliable2007} . Using the
similarity with the physical world, an atomic operation is
indivisible.
In the context of package management, we can talk about if a
package manager is atomic, if it can guarantee that a
transaction (for example, updating or installing a package)
is either completed or not, without any intermediate state.

With the analogy of memory for a program, leaving the
filesystem in a intermediate, inconsistent state is not
acceptable for reliable systems. The most basic example is a
package that is being updated, and some error occurs during
the transaction, such that some files have been written, and
some other not, as illustrated in figure
\ref{fig:atomicity}.


\begin{figure}[hbt]
    \centerfloat
    \includesvg[width=200pt]{assets/update_classic.svg}
    \caption{Failure during an update transaction for a
    classical \ac{PM}. Blue: files on the disk after the process.}
    \label{fig:atomicity}
\end{figure}

By having the package manager not rely on the mutation for
upgrades, then the problem of incomplete upgrades is solved
by the root. If mutation is not used, then the way to
upgrade a package is by using a different path on the
filesystem. The upgrade can operate on this different path
completely safely. Any interruption of the process will not
leave the original package in this inconsistent state.
Meanwhile, once the new package is in place, the update
operation is a matter of changing the references of the old
package to the new package.

For the implementation of this concept, in Android the way
to upgrade the system is via A/B updates. The system is partitioned
such that a new update doesn't mutate the existing system,
but rather copied into a different partition. When the
transfer is complete, the user is prompted to reboot the
system, and the bootloader will boot into the new partition.

